%% bigsposter.tex
%% Copyright 2017 Pietro Saccardi (saccardi at or.uni-bonn.de)
%
% This work may be distributed and/or modified under the
% conditions of the LaTeX Project Public License, either version 1.3
% of this license or (at your option) any later version.
% The latest version of this license is in
%   http://www.latex-project.org/lppl.txt
% and version 1.3 or later is part of all distributions of LaTeX
% version 2005/12/01 or later.
%
% This work has the LPPL maintenance status `maintained'.
%
% The Current Maintainer of this work is Pietro Saccardi.
%
% This work consists of the files bigsposter.cls, demo.tex, bigsposter.tex.

\documentclass[12pt, a4paper]{article}
\usepackage[utf8]{inputenc}
\usepackage[T1]{fontenc}
\usepackage{hyperref,listings,xcolor}
\usepackage[toc]{multitoc}
\renewcommand*{\multicolumntoc}{2}
\setlength{\columnseprule}{0pt}
\lstset{
  basicstyle=\small\ttfamily,
  basewidth=0.5em,
  language={[LaTeX]TeX},
  breaklines=true,
  frame=single,
  morekeywords={
    maketitle, blocktitle, advisor, blockbreak, posteryear, email,
    institute, blockfoot, blockcaption, includegraphics,
    blockfootnotemark, blockfootnotetext, blockfootnote, blockwidth,
    @unilogo, @bigslogo
  },
  keywordstyle=\color[rgb]{0,0,1},
  commentstyle=\color[rgb]{0.133,0.545,0.133},
  stringstyle=\color[rgb]{0.627,0.126,0.941},
  xleftmargin=1em,
  xrightmargin=1em
}

\title{The \texttt{bigsposter} class}
\date{June 23, 2017}
\author{Pietro Saccardi\\\href{mailto:saccardi@or.uni-bonn.de}{\texttt{saccardi@or.uni-bonn.de}}}

\newcommand{\cmdref}[2][\textbackslash{}]{\hyperref[h:#2]{\texttt{#1#2}}}
\newcommand{\pkgref}[1]{\href{https://www.ctan.org/pkg/#1}{\texttt{#1} package}}

\makeatletter
\let\origsubsection\subsection
\renewcommand\subsection{\@ifstar{\starsubsection}{\nostarsubsection}}

\newcommand\nostarsubsection[1]
{\subsectionprelude\origsubsection{#1}\subsectionpostlude}

\newcommand\starsubsection[1]
{\subsectionprelude\origsubsection*{#1}\subsectionpostlude}

\newcommand\subsectionprelude{%
  \vspace{\bigskipamount}
}

\newcommand\subsectionpostlude{\vspace{-0.5ex}}
\makeatother

\setcounter{tocdepth}{2}

\raggedbottom

\begin{document}
  \maketitle
  \begin{abstract}
    This class provides some basic environments to generate a PhD poster for a poster exhibition, and has been tailored on an old poster template for the BIGS program at the University of Bonn.
  \end{abstract}
  \vfill
  \begin{center}
    \textbf{\href{https://gitlab.com/5p4k/bigsposter}{\texttt{https://gitlab.com/5p4k/bigsposter}}}
  \end{center}
  \vfill
  \tableofcontents
\newpage
  \section{Getting started quickly}
    Copy-paste ready template, you just need to insert your personal information. This generates a simple 2-columns poster with a single title. Hopefully that's all you need to start.
    \begin{lstlisting}
\documentclass[a2paper]{bigsposter}

% Setup personal details
\author{Surname, Name}
\advisor{Prof.\ Dr.\ Advisor}
\title{The Title}
\posteryear{2017}
\email{yourname@institute.org}
\institute{Research Institute}

% These packages are optional (but you will use them)
\usepackage[utf8]{inputenc}
\usepackage[english]{babel}
\usepackage{amsmath, amsthm, amssymb}

\begin{document}
  \maketitle
  \begin{blockrow}
    \blocktitle{Title}
    Text goes here.
    \blockbreak
    \blockfoot
  \end{blockrow}
\end{document}
    \end{lstlisting}

  \section{Guide by example}
    The BIGS poster template consists of a \hyperref[h:titlefoot]{\emph{title header}}, and a several \emph{blocks} in the body, arranged in \hyperref[h:blockrow]{\emph{block rows}}. Ideally, it should consist of four blocks arranged in two rows, with row-major reading layout, in such a way that, when printed on a A2 paper sheet, each block is roughly an A4 page. The last block should include the \hyperref[h:blockfoot]{\emph{block footer}} and the bibliography. Each block may have one or more \hyperref[h:blocktitle]{\emph{block titles}}.

    This class also provides a minimal support for \hyperref[h:blockfootnote]{foot notes} and \hyperref[h:blockfigure]{figures} within the blocks.


    \subsection{Document class}
    \begin{lstlisting}
\documentclass[a2paper]{bigsposter} % or
\documentclass[a2paper, fixed]{bigsposter}
    \end{lstlisting}
    You are allowed to specify any paper size that is supported by the \pkgref{geometry}. The default size for BIGS poster is \texttt{a2paper}. The class already sets up the geometry with the following options: \texttt{margin=5mm, ignoreall}. All margins, font sizing and padding are set up relatively to the paper size.

    The \texttt{fixed} option is present only for retro-compatibility and should be used only in conjunction with \texttt{a2paper}. This sets all geometric parameters to the fixed values that were present in the old, original \LaTeX{} template; the parameters relative to the paper size generate slightly different placement for logo and header (although they can be used to generate posters in sizes other than A2).

    \subsection{Header and footer}\label{h:titlefoot}
    \begin{lstlisting}
\author{Surname, Name}
\advisor{Prof.\ Dr.\ Advisor}
\title{The Title}
\posteryear{2017}
\email{yourname@institute.org}
\institute{Research Institute}
% After \begin{document}
\maketitle
% Before the last \end{blockrow}
\blockfoot
    \end{lstlisting}
    You need to set up your details in the preamble for the title to be typeset correctly via \texttt{\textbackslash{}maketitle} and the footer via \cmdref{blockfoot}.

    \subsection{Typesetting blocks}
    \begin{lstlisting}
\begin{blockrow}
  \blocktitle{Title}
  Fist block
  \blockbreak
  Second block
\end{blockrow}
    \end{lstlisting}
    Use the \cmdref[]{blockrow} environment to open a new block row. Within it, blocks behave like columns; use \cmdref{blockbreak} to introduce a manual break, otherwise \LaTeX{} will balance the columns automatically. New titles can be issued with \cmdref{blocktitle}, also in the middle of the block. Within a block, you can access the width of the block with the macro \cmdref{blockwidth}.

    \subsection{Footnotes}
    \begin{lstlisting}
% ...
Some text inside a block row\blockfootnote{This will appear at the bottom.}.
    \end{lstlisting}
    The class defines an alternative footnote command, \cmdref{blockfootnote} which works inside a block row.

    \subsection{Figures and tables}
    \begin{lstlisting}
% Within a block row
\begin{blockfigure}
  \includegraphics{stuff.pdf}
  \blockcaption{Caption of this figure.}
\end{blockfigure}
\begin{blocktable}
  \begin{tabular}{c|c}
    one & two
  \end{tabular}
  \blockcaption{Caption of this table.}
\end{blocktable}
    \end{lstlisting}
    Within a block, use the \cmdref[]{blockfigure} environment to open a figure and \cmdref{blockcaption} to write out the caption. The \pkgref{caption} is being used internally, so you can freely customize the appearance of the caption with \texttt{\textbackslash{}captionsetup}. A similar environment is prebuilt for tables too. Use the \cmdref[]{blocktable} as a replacement for the standard \texttt{table} environment; \cmdref{blockcaption} is used to generate the caption of tables too.

    You may use foot notes in the caption, but you have to separate marker and text, using \cmdref{blockfootnotemark} and \cmdref{blockfootnotetext} as follows:
    \begin{lstlisting}
% ...
  \blockcaption{Caption of this figure\blockfootnotemark.}
\end{blockfigure}
\blockfootnotetext{Here goes the footnote.}
    \end{lstlisting}
\newpage
    \subsection{Minipages and side-by-side content}
    \begin{lstlisting}
% Inside a block row
\begin{minipage}[t]{0.3\blockwidth}
  Text on the left.
\end{minipage}
\begin{minipage}[t]{0.7\blockwidth}
  \begin{blockfigure}
    \includegraphics{stuff.pdf}
    \blockcaption{Figures can go side by side.}
  \end{blockfigure}
\end{minipage}
    \end{lstlisting}
    You can use the \texttt{minipage} environment to place side-to-side content within a single block. This is the main usage case for \cmdref{blockwidth}. Figures and footnotes would work perfectly inside a minipage; you may want to change the width of the caption in this case.

    It must be noted that minipage overrides \texttt{\textbackslash{}textwidth}, which is otherwise set to the overall width of all the blocks and their separators.

    \subsection{Bibliography}
    \begin{lstlisting}
% After the last \blockbreak
  \nocite{*} % If you want to include entries that are not cited
  \bibliography{samplebib} % samplebib.bib in the same folder
  \blockfoot
\end{blockrow}
    \end{lstlisting}
    You can use the usual bibliography within a block row. Place your bibliography right before \cmdref{blockfoot}.

  \section{Command reference}

    \subsection{\texttt{blockrow}}\label{h:blockrow}
    \begin{lstlisting}[frame=none, aboveskip=0pt, xleftmargin=0pt, xrightmargin=0pt]
\begin{blockrow}    % Default: two columns, or
\begin{blockrow}[4] % any number
    \end{lstlisting}
    Opens a multicolumn, framed environment. Takes one optional argument, defaulting to 2, which expresses the number of columns. Other \texttt{\textbackslash{}block}* commands are not guaranteed to work outside a block row. There are two counters affected by the creation of a block row, named respectively \texttt{blockrow} and \texttt{block} (the meaning is obvious).

    \subsection{\texttt{blockbreak}}\label{h:blockbreak}
    \begin{lstlisting}[frame=none, aboveskip=0pt, xleftmargin=0pt, xrightmargin=0pt]
\blockbreak % No argument
    \end{lstlisting}
    Opens a new block, increases the \texttt{block} counter; takes no argument. Valid only within a block row. Beware of empty lines around this command (issue \ref{issue:whitespace}).

    \subsection{\texttt{blocktitle}}\label{h:blocktitle}
    \begin{lstlisting}[frame=none, aboveskip=0pt, xleftmargin=0pt, xrightmargin=0pt]
\blocktitle{Title} % One mandatory argument
    \end{lstlisting}
    Typesets a new title, valid only within a block row; it does not have a corresponding section level like standard \TeX{} commands, it is just a different aesthetic typeset. Beware of empty lines around this command (issue \ref{issue:whitespace}).

    \subsection{\texttt{blockfoot}}\label{h:blockfoot}
    \begin{lstlisting}[frame=none, aboveskip=0pt, xleftmargin=0pt, xrightmargin=0pt]
\blockfoot % No argument
    \end{lstlisting}
    Valid only within a block row, typesets a ruler followed by year, institute and email. It should be the last command within the last block row.

    \subsection{\texttt{blockwidth}}\label{h:blockwidth}
    \begin{lstlisting}[frame=none, aboveskip=0pt, xleftmargin=0pt, xrightmargin=0pt]
\blockwidth % A length macro
    \end{lstlisting}
    A length macro that can be used for sizing. Valid only within a block row; it expresses the width of the text area in a block (i.e.\ the effective width after subtracting borders and padding of a single column).

    \subsection{\texttt{blockfigure}}\label{h:blockfigure}
    \begin{lstlisting}[frame=none, aboveskip=0pt, xleftmargin=0pt, xrightmargin=0pt]
\begin{blockfigure}    % No argument
    \end{lstlisting}
    Just a glorified \texttt{center} enviroment with a bit less vertical space.

    \subsection{\texttt{blocktable}}\label{h:blocktable}
    \begin{lstlisting}[frame=none, aboveskip=0pt, xleftmargin=0pt, xrightmargin=0pt]
\begin{blocktable}    % No argument
    \end{lstlisting}
    Just a glorified \texttt{center} enviroment with a bit less vertical space.

    \subsection{\texttt{blockcaption}}\label{h:blockcaption}
    \begin{lstlisting}[frame=none, aboveskip=0pt, xleftmargin=0pt, xrightmargin=0pt]
\blockcaption{Caption}                 % One mandatory argument.
\blockcaption[0.9\blockwidth]{Caption} % One optional argument, the width of the caption.
    \end{lstlisting}
    Typesets the caption of a figure, to be used within the \cmdref[]{blockfigure} and \cmdref[]{blocktable} environment. Takes one mandatory argument, the caption content, and one optional argument, the width of the paragraph box of the caption, which defaults to \texttt{0.8\textbackslash{}blockwidth}.
    Within a caption, \cmdref{blockfootnote} cannot be used, it must be broken into \cmdref{blockfootnotemark} (inside the caption) and \cmdref{blockfootnotetext} (outside the figure).

    \subsection{\texttt{blockfootnotemark}}\label{h:blockfootnotemark}
    \begin{lstlisting}[frame=none, aboveskip=0pt, xleftmargin=0pt, xrightmargin=0pt]
\blockfootnotemark % No argument
    \end{lstlisting}
    Typesets the superscript number which marks a new foot note.

    \subsection{\texttt{blockfootnotetext}}\label{h:blockfootnotetext}
    \begin{lstlisting}[frame=none, aboveskip=0pt, xleftmargin=0pt, xrightmargin=0pt]
\blockfootnotetext{Foot note content} % One mandatory argument
    \end{lstlisting}
    Typesets the foot note text of the last \cmdref{blockfootnotemark} invoked.

    \subsection{\texttt{blockfootnote}}\label{h:blockfootnote}
    \begin{lstlisting}[frame=none, aboveskip=0pt, xleftmargin=0pt, xrightmargin=0pt]
\blockfootnote{Foot note content} % One mandatory argument
    \end{lstlisting}
    Combines \cmdref{blockfootnotemark} and \cmdref{blockfootnotetext}.

    \subsection{Special drawing commands \texttt{@unilogo} and \texttt{@bigslogo}}
    \begin{lstlisting}[frame=none, aboveskip=0pt, xleftmargin=0pt, xrightmargin=0pt]
\makeatletter
\@unilogo{5em}  % One length mandatory argument
\@bigslogo{8em} % One length mandatory argument
\makeatother
    \end{lstlisting}
    These two command typeset in Ti\textit{k}Z the logo of the University of Bonn and BIGS program, respectively, automatically scaled to have the width specified in the unique argument. No padding is added.
\bigskip\bigskip
  \section{Technical notes}
    The need for overriding footnotes, figures, tables and captions arises from the fact that the core environment that provides a block row is made using the \texttt{framed} and \texttt{multicols} environment, in which these objects do not work. In other words: floats do not work, they are just simulated.
    \subsection{Used packages and classes}
    \begin{itemize}\setlength{\itemindent}{4em}
      \item[\texttt{article}] as a base class.
      \item[\texttt{tikz}] for typesetting logos.
      \item[\texttt{geometry}] for setting up page layout.
      \item[\texttt{parskip}] to skip white space instead of using indentation in the text body.
      \item[\texttt{xcolor}] for typesetting logos and titles.
      \item[\texttt{lmodern}] to suppress missing symbol warnings on the custom sized font title.
      \item[\texttt{framed}] to generate a robust frame around each block row.
      \item[\texttt{hyperref}] for linking the email.
      \item[\texttt{multicol}] to produce the subdivision between blocks.
      \item[\texttt{caption}] to produce captions inside a different figure environment.
    \end{itemize}
    \subsection{Known issues and limitations}
      \subsubsection{Whitespace around block breaks and titles} \label{issue:whitespace} The \cmdref{blocktitle} and \cmdref{blockbreak} are not completely tolerant w.r.t.\ new paragraphs and new lines in their immediate surroundings. Normally a sectioning command starts a new paragraph if needed, and does not add up space if it is on top of a new page. It is not the case with these two commands; if the user accidentally typesets a new paragraph, it may add some extra whitespace before or after a block break or a title, even if at the beginning of a new block.
      \subsubsection{Vertical spacing in figures} When a block figure is started inside a minipage, the vertical spacing seem to be broken; the content is shifted slightly out of the minipage, and there is no vertical spacing between the content and the possible caption.
      \subsubsection{Footnote limitation} Foot notes do not support custom markers.
\end{document}
